\section{Datasets \& Experimental Setup}

\subsection{Copernicus Sentinels: 1 and 2}

We use paired observations from the Copernicus Sentinel-1 and Sentinel-2 missions. Sentinel-1 provides C-band SAR data in Interferometric Wide Swath (IW) mode with dual polarisation (VV+VH), while Sentinel-2 delivers multispectral optical imagery with 13 bands at spatial resolutions of 10–60~m. These complementary modalities form the basis for SAR-to-optical translation.

The dataset was collected from [\textit{source}] over [\textit{region}] during [\textit{timeframe}]. Preprocessing included radiometric calibration, speckle filtering, and coregistration of Sentinel-1 with Sentinel-2, followed by resampling to a common resolution, cloud masking, and band-wise normalisation. In total, [\textit{number}] paired samples were extracted and split into [\textit{train/val/test split}].

\subsection{Implementation and Training Configuration}
We implement the BiMBU model in PyTorch, leveraging GPU-optimized operations for its \texttt{Mamba} and \texttt{Conv2D} modules to ensure computational efficiency. For reproducability, we release the complete source code for our implementation in a public repository~\cite{repo}. For training and inference, we use an AI-server with a single NVIDIA A100 80GB GPU with 64 CPU cores and 512 GB RAM, using CUDA 12.4 and PyTorch 2.6.0+cu124. BiMBU and baseline models were trained with a batch size of 128 for up to 150 epochs, using early stopping with patience of 15 and saving the best model based on validation loss. We use Adam optimizer with a learning rate of $10^{-4}$ and \texttt{ReduceLROnPlateau} scheduler (patience: 7, factor: 0.5).

\subsection{Evaluation Metrics}
We employ a set of metrics to evaluate reconstruction accuracy from multiple perspectives. 

Pixel-level fidelity is assessed using Mean Absolute Error (MAE) and Root Mean Squared Error (RMSE), which quantify absolute and squared deviations, respectively. Peak Signal-to-Noise Ratio (PSNR) measures the ratio between signal and reconstruction error, while the Structural Similarity Index (SSIM) captures perceived structural similarity in spatial patterns. 

To evaluate spectral consistency, Spectral Angle Mapper (SAM) is used to measure the angular deviation between predicted and reference spectral signatures, and Spectral Information Divergence (SID) to compare their probabilistic distributions. 

We assess downstream utility by computing errors on remote sensing indices derived from reconstructed bands: the Normalized Difference Vegetation Index (NDVI), the Normalized Difference Water Index (NDWI), and the Normalized Burn Ratio (NBR). 

Per-band metrics (e.g., SSIM, MAE, RMSE per Sentinel-2 band) are reported to highlight reconstruction quality at the individual spectral channel level. All metrics are computed after reversing the normalization to the original physical scale.

