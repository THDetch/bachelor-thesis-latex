\chapter{Conclusion}
This thesis explored the potential of SAR-to-optical image translation as a generative framework for synthesizing multispectral optical imagery from radar data, aiming to mitigate the limitations of cloud-contaminated optical remote sensing. Using co-registered Sentinel-1 and Sentinel-2 data from the SEN12-MS dataset, the Pix2Pix conditional generative adversarial network (cGAN) was trained to translate dual-polarized SAR inputs into full-spectrum optical outputs across all 13 Sentinel-2 bands.
The study confirmed that meaningful spectral and spatial correlations exist between the SAR and optical domains, enabling reliable reconstruction of high-fidelity optical imagery. Reconstruction quality was found to vary across spectral bands, reflecting differences in wavelength sensitivity and signal characteristics. The model’s performance in cloud removal further demonstrated that SAR-to-optical translation can effectively generate cloud-free imagery even without explicit training for this task—achieving results comparable to or surpassing several state-of-the-art approaches.

Ablation experiments showed that combining SSIM and LPIPS losses with the standard GAN objective enhances reconstruction consistency and perceptual realism. Meanwhile, analysis of per-band clipping highlighted a commonly overlooked limitation in existing literature: the uniform clipping of optical data across all bands disregards their distinct value distributions, potentially reducing spectral fidelity.
While the results validate the feasibility and promise of SAR-to-optical translation, challenges remain. Model performance decreases in textureless or spectrally complex regions, and temporal generalization is limited by training exclusively on winter data. Addressing these limitations through seasonally diverse training sets, diffusion-based generative models, and per-band-aware preprocessing would likely further improve spectral realism and robustness.

Overall, this work establishes that generative models—particularly GAN-based architectures—can reconstruct full-spectrum, cloud-free optical imagery from SAR data with competitive accuracy. These findings reinforce the potential of generative approaches to enhance the temporal continuity and usability of optical remote sensing data under challenging observation conditions.
