\chapter{Discussion}
\label{chapter:discussion}
This thesis successfully achieved the objectives outlined at the beginning. The first goal, which is reconstructing full-spectrum optical images solely from SAR inputs, was met with remarkable results after fine-tuning, despite the inherent challenges of training GANs. The observed difference in performance when training the Pix2Pix model on 20\% versus 100\% of the dataset highlights the computational intensity and data dependency of GAN-based approaches.
Although the findings confirm the capability of GANs for this task, the model exhibits limitations when the SAR input lacks structural detail or sufficient backscatter information, leading to noise-like outputs. In this context, diffusion models emerge as a promising alternative for future research, offering greater stability and generative fidelity.

Furthermore, the thesis demonstrated the model’s ability to address the cloud cover problem using samples from the SEN12-MS-CR dataset, despite the model not being explicitly trained for this task. Remarkably, the Pix2Pix model even surpassed the state-of-the-art DiffCR model—an unexpected outcome, given that Pix2Pix is typically reported in the literature to produce weaker results.
Because the model was trained exclusively to translate SAR data into cloud-free optical imagery, cloud thickness or density did not appear to influence the reconstruction process. This suggests that SAR-to-optical translation might, in fact, be more effective for cloud removal than approaches combining SAR with cloud-contaminated optical images, as the latter can introduce spectral noise due to temporal discrepancies between acquisitions. A comparative evaluation of both strategies therefore remains an important avenue for future work.

The per-band evaluation indicated that model performance is largely independent of spectral characteristics. Excluding the 60\,m bands—and thereby reducing problem complexity and computational cost—did not yield a substantial improvement. Nonetheless, training models on only the spectral composites relevant to specific downstream tasks could be advantageous, since most Earth observation applications employing cloud removal as a preprocessing step do not require all 13 Sentinel-2 bands~\cite{CR_SEN2_dRNN}.
Regarding generalizability, the random global sampling of the SEN12 dataset family ensures reasonable spatial robustness. However, fine-tuning the model on local subsets may be beneficial for specific applications. Temporally, the model’s limitation is acknowledged: it was trained exclusively on winter data, and although results across different meteorological seasons were analyzed, a more balanced dataset—sampled across all four seasons—would likely improve temporal generalizability.

From the ablation study on loss functions, it is recommended to combine pixel-wise reconstruction losses with perceptual components such as LPIPS, which help capture perceptual similarity beyond pixel accuracy. Moreover, incorporating spectral-aware metrics such as the Spectral Angle Mapper (SAM) into the loss function could enhance spectral consistency across bands—an aspect worth exploring in future research.
Another aspect that is totally neglected in the literature is the per-band clipping, where the studies clip the optical value ranges to [0, 10000], neglecteing the different values ranges between the individual bands. The data analysis given in this thesis should direct future work in that direction.

Finally, this work highlights an overlooked aspect in the literature: the per-band clipping of optical values. All previous studies uniformly clip optical data to the range [0, 10,000], disregarding the distinct value distributions among spectral bands. The statistical analysis presented in this thesis underscores the importance of adopting band-aware clipping strategies in future work to better preserve spectral integrity.