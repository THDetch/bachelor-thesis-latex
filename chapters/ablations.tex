\chapter{Ablation Studies}
\section{Effect of Loss Functions}
\label{subsec:ablation_loss}

To evaluate the contribution of each loss component to the overall model performance, an ablation study was conducted. Four training configurations were compared:
\begin{enumerate}
    \item $\mathcal{L}_{\text{GAN}} + \mathcal{L}_{\text{L1}}$,
    \item $\mathcal{L}_{\text{GAN}} + \mathcal{L}_{\text{L1}} + \mathcal{L}_{\text{SSIM}}$,
    \item $\mathcal{L}_{\text{GAN}} + \mathcal{L}_{\text{L1}} + \mathcal{L}_{\text{LPIPS}}$, and
    \item the full combination $\mathcal{L}_{\text{GAN}} + \mathcal{L}_{\text{L1}} + \mathcal{L}_{\text{SSIM}} + \mathcal{L}_{\text{LPIPS}}$.
\end{enumerate}
This analysis aimed to isolate the contribution of each additional loss term to both quantitative performance and visual reconstruction quality. The evaluation was conducted using the same IQA metrics employed throughout the thesis, namely SSIM, PSNR, LPIPS, SAM, MAE, and RMSE. Several training loops were conducted using the same preprocessing procedure described in Section~\ref{subsec:preprocessing}. All models were trained under identical settings, hyperparameters, datasets, and number of epochs to ensure a fair and consistent comparison across the different loss configurations.

Table~\ref{tab:ablation_quantitative} summarizes the quantitative performance across the four loss configurations. As shown, the baseline configuration without SSIM and LPIPS achieved the lowest performance across most metrics, with the exception of PSNR. This configuration also represents the weakest setup in terms of overall image quality. As illustrated in Figure~\ref{fig:ablation_samples}(b), images generated by the baseline model differ significantly from the ground truth, both texturally and perceptually. 

\begin{table}[h!]
    \centering
    \resizebox{\textwidth}{!}{%
        \begin{tabular}{lcccccc}
            \toprule
            \textbf{Loss Configuration}                                                                                   & \textbf{SSIM $\uparrow$} & \textbf{PSNR (dB) $\uparrow$} & \textbf{LPIPS $\downarrow$} & \textbf{SAM~(°) $\downarrow$} & \textbf{MAE $\downarrow$} & \textbf{RMSE $\downarrow$} \\
            \midrule
            $\mathcal{L}_{\text{GAN}} + \mathcal{L}_{\text{L1}}$                                                          & 0.820                    & 26.38                         & 0.287                       & 7.88                          & 229                       & 441                        \\
            $\mathcal{L}_{\text{GAN}} + \mathcal{L}_{\text{L1}} + \mathcal{L}_{\text{SSIM}}$                              & \textbf{0.862}           & \textbf{27.67}                & 0.399                       & \textbf{6.67}                 & 198                       & \textbf{380}               \\
            $\mathcal{L}_{\text{GAN}} + \mathcal{L}_{\text{L1}} + \mathcal{L}_{\text{LPIPS}}$                             & 0.842                    & 27.58                         & \textbf{0.213}              & 7.05                         & 201                       & 385                        \\
            $\mathcal{L}_{\text{GAN}} + \mathcal{L}_{\text{L1}} + \mathcal{L}_{\text{SSIM}} + \mathcal{L}_{\text{LPIPS}}$ & 0.859                    & 27.65                         & 0.224                       &  6.71                          & \textbf{195}              & 382                        \\
            \bottomrule
        \end{tabular}%
    }
    \caption[Quantitative ablation study across loss configurations]{Quantitative results of the ablation study across different loss configurations. Best values per metric are shown in bold.}
    \label{tab:ablation_quantitative}
\end{table}

Notably, integrating SSIM alone yielded the highest quantitative scores in several metrics. However, the qualitative results under this configuration reveal perceptual inconsistencies and reduced visual realism. This discrepancy arises because SSIM does not always align with human perceptual judgments of image similarity. As reported by NVIDIA in~\cite{nvidia_Understanding_SSIM}, SSIM can overemphasize small intensity variations in dark regions, overlook significant color shifts, and assign high similarity near edges even when visible artifacts are present.

\begin{figure}[h!]
    \centering
    \setlength{\tabcolsep}{2pt} % horizontal padding between columns
    \renewcommand{\arraystretch}{1.0} % vertical padding

    % Adjust width so that 6 images fit one row across the text width
    % (tweak 0.155\textwidth to 0.158 or 0.152 if needed)
    \begin{tabular}{*{6}{c}}
        % \toprule
        (a) & (b) & (c) & (d) & (e) & (f) \\
        % \midrule

        % ------------------- Row 1 -------------------
        \includegraphics[width=0.155\textwidth]{img/ablation/sample_1/sar.png}   &
        \includegraphics[width=0.155\textwidth]{img/ablation/sample_1/none.png}  &
        \includegraphics[width=0.155\textwidth]{img/ablation/sample_1/ssim.png}  &
        \includegraphics[width=0.155\textwidth]{img/ablation/sample_1/lpips.png} &
        \includegraphics[width=0.155\textwidth]{img/ablation/sample_1/all.png}   &
        \includegraphics[width=0.155\textwidth]{img/ablation/sample_1/gt.png}                                  \\
        % ------------------- Row 2 -------------------
        \includegraphics[width=0.155\textwidth]{img/ablation/sample_2/sar.png}   &
        \includegraphics[width=0.155\textwidth]{img/ablation/sample_2/none.png}  &
        \includegraphics[width=0.155\textwidth]{img/ablation/sample_2/ssim.png}  &
        \includegraphics[width=0.155\textwidth]{img/ablation/sample_2/lpips.png} &
        \includegraphics[width=0.155\textwidth]{img/ablation/sample_2/all.png}   &
        \includegraphics[width=0.155\textwidth]{img/ablation/sample_2/gt.png}                                  \\
        % ------------------- Row 3 -------------------
        \includegraphics[width=0.155\textwidth]{img/ablation/sample_3/sar.png}   &
        \includegraphics[width=0.155\textwidth]{img/ablation/sample_3/none.png}  &
        \includegraphics[width=0.155\textwidth]{img/ablation/sample_3/ssim.png}  &
        \includegraphics[width=0.155\textwidth]{img/ablation/sample_3/lpips.png} &
        \includegraphics[width=0.155\textwidth]{img/ablation/sample_3/all.png}   &
        \includegraphics[width=0.155\textwidth]{img/ablation/sample_3/gt.png}                                  \\
        % ------------------- Row 4 -------------------
        \includegraphics[width=0.155\textwidth]{img/ablation/sample_4/sar.png}   &
        \includegraphics[width=0.155\textwidth]{img/ablation/sample_4/none.png}  &
        \includegraphics[width=0.155\textwidth]{img/ablation/sample_4/ssim.png}  &
        \includegraphics[width=0.155\textwidth]{img/ablation/sample_4/lpips.png} &
        \includegraphics[width=0.155\textwidth]{img/ablation/sample_4/all.png}   &
        \includegraphics[width=0.155\textwidth]{img/ablation/sample_4/gt.png}                                  \\
        % \bottomrule
    \end{tabular}

    \caption[Qualitative ablation study across loss configurations]{%
    Qualitative ablation results showing representative samples (rows) under four configurations (columns):\\
    \textbf{(a)}~Input SAR (pseudo RGB)\\
    \textbf{(b)}~$\mathcal{L}_{\text{GAN}}{+}\mathcal{L}_{\text{L1}}$\\
    \textbf{(c)}~$\mathcal{L}_{\text{GAN}}{+}\mathcal{L}_{\text{L1}}{+}\mathcal{L}_{\text{SSIM}}$\\
    \textbf{(d)}~$\mathcal{L}_{\text{GAN}}{+}\mathcal{L}_{\text{L1}}{+}\mathcal{L}_{\text{LPIPS}}$\\
    \textbf{(e)}~all four losses combined\\
    \textbf{(f)}~ground-truth Sentinel-2 optical image (RGB bands).
    }
    \label{fig:ablation_samples}
\end{figure}

Since the primary objective of LPIPS is to measure perceptual similarity, incorporating it into the baseline configuration led to notable improvements in the preservation of low-level features. As illustrated in column~(d) of Figure~\ref{fig:ablation_samples}, the model was able to maintain sharper edges and more distinct boundaries. Moreover, the generated images appear visually more realistic and closely resemble the ground truth in texture and detail. However, the colour reproduction achieved with LPIPS is slightly inferior to that obtained using SSIM, as the SSIM-based model produces more natural and accurate colours overall.

The full combination of all four losses achieved the best balance between pixel-level accuracy, perceptual realism, and structural coherence. 
Its SAM values were the second best, differing only marginally from the SSIM-only configuration. 
By incorporating both SSIM and LPIPS, the model attains an effective balance between color, luminance, and perceptual realism, as shown in column~(e) of Figure~\ref{fig:ablation_samples}. 
Similar findings were also reported by~\cite{s2o_ViT_cGAN} and~\cite{CR_RS_GAN_s2o}. 
 
Overall, the results demonstrate the incremental benefit of incorporating perceptual and structural similarity terms alongside the adversarial $\mathcal{L}_{\text{GAN}}$ and $\mathrm{L1}$ objectives. 

\section{Effect of Excluding 60 m Bands}
Given the varying data distributions and the originally different spatial resolutions of the Sentinel-2 bands (even though all were resampled to 10m in the dataset), it was hypothesized that including the 60m bands might introduce noise and confuse the model during training. To examine this assumption, an ablation study was conducted using the same training configuration as for the full 13-band model, both trained on 20\% of the winter subset.

Surprisingly, removing the 60m bands did not lead to any improvement. Instead, both SSIM and LPIPS scores decreased, as reported in Table~\ref{tab:ablation_excluding_60m}. This suggests that the 60m bands, which primarily serve for atmospheric correction (e.g., B1, B9, and B10), provide complementary spectral information that contributes to maintaining spectral fidelity during training.

\begin{table}[h!]
    \centering
    \setlength{\tabcolsep}{8pt}
    \renewcommand{\arraystretch}{1.15}
    \caption[Overall performance when excluding 60m bands]{Overall performance comparing all 13 bands vs. excluding the 60m bands (B1, B9, B10).}
    \label{tab:ablation_excluding_60m}
    \begin{tabular}{lccc}
        \hline
        \textbf{Setting} & \textbf{SSIM $\uparrow$} & \textbf{PSNR (dB) $\uparrow$} & \textbf{SAM~(°) $\downarrow$} \\
        \hline
        All 13 bands & \textbf{0.859390} & \textbf{27.650251} & 6.7064 \\
        Excluding 60m (B1, B9, B10) & 0.825861 & 26.474423 & \textbf{6.3339} \\
        \hline
    \end{tabular}
    % Note: SAM is computed in 13-band space for "All 13 bands" and in 10-band space when excluding 60 m bands.
\end{table}

Interestingly, while the SAM slightly improved after excluding the 60m bands, this does not necessarily contradict the decline in PSNR and SSIM. The model may have learned to reproduce more spectrally consistent relationships between the remaining bands (hence a smaller spectral angle), yet failed to preserve the absolute reflectance magnitudes and spatial details, leading to lower overall reconstruction quality. This indicates a trade-off between radiometric fidelity and spectral coherence in the absence of the full spectral range. For qualitative evaluation, Figure~\ref{fig:ablation_excluding_60m_qualitative} illustrates the generated optical images obtained from models trained with the full spectral configuration and with the 60,m bands excluded.

Similarly, the per-band evaluation (Table~\ref{tab:ablation_excluding_60m_per_band}) confirms this trend. For most bands, SSIM remains stable when all bands are included, indicating consistent structural reconstruction quality. However, PSNR decreases for nearly all bands when the 60m bands are excluded, with the exception of B1, demonstrating that the full spectral configuration better supports radiometric reconstruction.
\begin{figure}[h!]
    \centering
    \setlength{\tabcolsep}{2pt} % horizontal padding between columns
    \renewcommand{\arraystretch}{1.0} % vertical padding

    \begin{tabular}{cccc}
        % ------------------- Rows -------------------
        (a) & (b) & (c) & (d) \\
        \includegraphics[width=0.2\textwidth, height=0.2\textheight, keepaspectratio]{img/exclusion_60_m/bands10/sample_000034_sar_pseudo.png} &
        \includegraphics[width=0.2\textwidth, height=0.2\textheight, keepaspectratio]{img/exclusion_60_m/bands10/sample_000034_pred_rgb.png} &
        \includegraphics[width=0.2\textwidth, height=0.2\textheight, keepaspectratio]{img/exclusion_60_m/bands13/sample_000034_pred_rgb.png} &
        \includegraphics[width=0.2\textwidth, height=0.2\textheight, keepaspectratio]{img/exclusion_60_m/bands10/sample_000034_true_rgb.png} \\

        \includegraphics[width=0.2\textwidth, height=0.2\textheight, keepaspectratio]{img/exclusion_60_m/bands10/sample_000050_sar_pseudo.png} &
        \includegraphics[width=0.2\textwidth, height=0.2\textheight, keepaspectratio]{img/exclusion_60_m/bands10/sample_000050_pred_rgb.png} &
        \includegraphics[width=0.2\textwidth, height=0.2\textheight, keepaspectratio]{img/exclusion_60_m/bands13/sample_000050_pred_rgb.png} &
        \includegraphics[width=0.2\textwidth, height=0.2\textheight, keepaspectratio]{img/exclusion_60_m/bands10/sample_000050_true_rgb.png} \\

        \includegraphics[width=0.2\textwidth, height=0.2\textheight, keepaspectratio]{img/exclusion_60_m/bands10/sample_000010_sar_pseudo.png} &
        \includegraphics[width=0.2\textwidth, height=0.2\textheight, keepaspectratio]{img/exclusion_60_m/bands10/sample_000010_pred_rgb.png} &
        \includegraphics[width=0.2\textwidth, height=0.2\textheight, keepaspectratio]{img/exclusion_60_m/bands13/sample_000010_pred_rgb.png} &
        \includegraphics[width=0.2\textwidth, height=0.2\textheight, keepaspectratio]{img/exclusion_60_m/bands10/sample_000010_true_rgb.png} \\

        \includegraphics[width=0.2\textwidth, height=0.2\textheight, keepaspectratio]{img/exclusion_60_m/bands10/sample_000065_sar_pseudo.png} &
        \includegraphics[width=0.2\textwidth, height=0.2\textheight, keepaspectratio]{img/exclusion_60_m/bands10/sample_000065_pred_rgb.png} &
        \includegraphics[width=0.2\textwidth, height=0.2\textheight, keepaspectratio]{img/exclusion_60_m/bands13/sample_000065_pred_rgb.png} &
        \includegraphics[width=0.2\textwidth, height=0.2\textheight, keepaspectratio]{img/exclusion_60_m/bands10/sample_000065_true_rgb.png} \\
    \end{tabular}

    \caption[Qualitative results when excluding 60\,m bands]{%
    Qualitative comparison of generated optical images when excluding the 60\,m Sentinel-2 bands. 
    Columns: \textbf{(a)}~SAR input (pseudo-RGB), 
    \textbf{(b)}~generated optical image trained without the 60\,m bands, 
    \textbf{(c)}~generated optical image trained with all 13 bands, and 
    \textbf{(d)}~reference cloud-free Sentinel-2 image.}
    \label{fig:ablation_excluding_60m_qualitative}
\end{figure}

\begin{table}[h!]
    \centering
    \setlength{\tabcolsep}{6pt}
    \renewcommand{\arraystretch}{1.15}
    \caption[Per-band performance when excluding 60m bands]{Per-band performance when excluding 60m bands.}
    \label{tab:ablation_excluding_60m_per_band}
    \begin{tabular}{lcccc}
    \hline
    \textbf{Band} & \textbf{SSIM (no 60\,m)} $\uparrow$ & \textbf{SSIM (13)} $\uparrow$ & \textbf{PSNR (no 60\,m) [dB]} $\uparrow$ & \textbf{PSNR (13) [dB]} $\uparrow$ \\
    \hline
    B1   & ---    & 0.9634 & ---    & 29.749 \\
    B2   & 0.9431 & 0.9430 & 34.151 & 34.057 \\
    B3   & 0.9064 & 0.9060 & 31.691 & 31.718 \\
    B4   & 0.8348 & 0.8340 & 28.039 & 28.145 \\
    B5   & 0.8707 & 0.8729 & 28.098 & 28.257 \\
    B6   & 0.8199 & 0.8231 & 26.740 & 26.926 \\
    B7   & 0.7890 & 0.7925 & 25.694 & 25.884 \\
    B8   & 0.7375 & 0.7416 & 25.534 & 25.739 \\
    B8A  & 0.7687 & 0.7722 & 25.095 & 25.311 \\
    B9   & ---    & 0.8666 & ---    & 24.617 \\
    B10  & ---    & 0.8826 & ---    & 24.571 \\
    B11  & 0.7774 & 0.7809 & 23.395 & 23.727 \\
    B12  & 0.8034 & 0.8073 & 24.956 & 25.263 \\
    \hline
    \end{tabular}
\end{table}


% \begin{table}[h!]
% \centering
% \caption[Quantitative results across data scales and seasonal subsets]{%
% Quantitative performance of the \textit{pix2pix} model across different training data scales and seasonal subsets of the SEN12-MS dataset. Arrows ($\uparrow$ / $\downarrow$) indicate whether higher or lower values denote better performance, respectively.}
% \begin{tabular}{lcccc}
% \toprule
% \textbf{Dataset} & \textbf{SSIM $\uparrow$} & \textbf{PSNR (dB) $\uparrow$} & \textbf{LPIPS $\downarrow$} & \textbf{SAM (°) $\downarrow$} \\
% \midrule
% Winter & \textbf{0.888} & \textbf{32.63} & \textbf{0.173} & \textbf{4.41} \\
% Spring & 0.791 & 20.00 & 0.297 & 12.06  \\
% Summer & 0.800 & 22.79 & 0.268 & 12.43  \\
% Fall & 0.788 & 22.40 & 0.269 & 13.67  \\
% \bottomrule
% \end{tabular}
% \label{tab:quantitative_results_all}
% \end{table}
